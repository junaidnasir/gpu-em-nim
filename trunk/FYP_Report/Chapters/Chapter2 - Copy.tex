% Chapter Template

\chapter{Related Work} % Main chapter title

\label{Chapter2} % Change X to a consecutive number; for referencing this chapter elsewhere, use \ref{ChapterX}

\lhead{Chapter 2. \emph{Related}} % Change X to a consecutive number; this is for the header on each page - perhaps a shortened title

%----------------------------------------------------------------------------------------
%	SECTION 1
%----------------------------------------------------------------------------------------

\section{Related Work}

We consider the large body of testbed,
    for both sensor and traditional networking,
    as related work to this work.

With arguments quite similar to our work,
    recent efforts have also established underwater testbeds~\citep{Peng,aqua-tune,Freitag}.
Aqua-Lab \citep{Peng},
	provides a remotely accessible underwater acoustic testbed,
	to emulate algorithms and protocols designed for underwater
	sensor networks, but for static nodes positioning in an aquarium
	of dimensions 2m$\times$1m$\times$1m.
Similarly,
	Aqua-TUNE ~\citep{aqua-tune} deploys a
	testbed in a lake,
    with nodes deployed using bouys having an AP-like network
    through which protocol parameters can be updated.
Freitag et al, have established
	remotely accessible testbeds at three different environmental location,
    allowing users to perform experiments	
	in deep and shallow water~\citep{Freitag}.
However nodes have to be
    deployed using boats, making
    the process somewhat cumbersome.
While these last two efforts provide a greater
    diversity in deployment environment,
    they suffer from being difficult to access
    and having limited experimental
    duration due to usage of battery packs.
Furthermore, all existing testbeds do not provide
    a programmable way to perform range-based
    experiments --- something quite necessary for
    network and phy-layer protocol evaluation.


The programmability of our testbed is
    influenced by traditional networking testbeds such
     as DETER~\citep{Deter}
 	and Emulab Lab~\citep{Emulab}.
These testbeds are designed to allow networking researchers
    a rich and customizable  networking environment
    to run detailed experiments.
Deterlab provides a platform for
	development, repeatable experimentation, and testing of
	new cyber-security technologies.
Emulab is the seminal
 	network testbed, and provides researchers with a wide
	range of environments where they can develop, debug,
 	and evaluate networking protocols
 and distributed systems under \emph{controlled} delay
 and congestion scenarios.
This remote accessibility,
    flexibility, and replicable nature
    has made these testbed quite famous and useful
    in the academic environment..
