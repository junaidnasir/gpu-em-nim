% Chapter Template

\chapter{Design and Implementation} % Main chapter title

\label{Chapter2} % Change X to a consecutive number; for referencing this chapter elsewhere, use \ref{ChapterX}

\lhead{Chapter 2. \emph{Design and Implementation}} % Change X to a consecutive number; this is for the header on each page - perhaps a shortened title

%----------------------------------------------------------------------------------------
%	SECTION 1
%----------------------------------------------------------------------------------------

\section{Modeling of a denser slab}

A denser slab having $\mu_r = 2\mu_o$ and $\epsilon_r = 2\epsilon_o$
is simulated to test the results and accuracy of our algorithm. these results will be compared with
results of NIM slab.

%-----------------------------------
%	SUBSECTION 1.1
%-----------------------------------
\subsection{Finite Difference Time Domain (FDTD) technique}
FDTD method is used to solve problems related to electromagnetic. It is very easy to implement but require much computational power and time as this is a recursive technique. The main benefit of using this technique is that it is accurate on wide range of frequency. 

Maxwell's equations (Ampere's and Faraday's laws) can be written as finite differences using FDTD. A function $f(x)$ whose value need to be found at $x_o$ is given by %ref here
	\begin{equation}
	\left. \frac {df(x)}{d(x)}\right|_{x=x_o} \approx 
	\frac {f \left( x_o + \frac{\delta}{2} \right) - f \left( x_o - \frac{\delta}{2} \right) }{\delta}
	\end{equation}
this equation provides an approximation of derivative of $f(x)$ at $x_o$ but the function is sampled at an offset $\delta$ from the original point $x_o$. It have second-order accuracy or second-order behavior. This implies if $\delta$ is reduced by a factor of 10 the error will be reduced by 100. if $\delta = 0$ then $error = 0$.
%-----------------------------------
%	SUBSECTION 1.2
%-----------------------------------
\subsection{The Yee Algorithm}
Kane Yee proposed FDTD algorithm in 1966% ref here
. It can be summarized as follows:
\begin{enumerate}
\item Differential forms of Maxwell's equations are written with finite differences.
\item Solve the difference equations to get "update equations" that express the future value in terms of past value. Figure \ref{fig:fdtd}
\begin{figure}[htbp]
	\centering
		\includegraphics[width=4in]{Pictures/fdtd.jpg}
		%\rule{35em}{0.5pt}
	\caption[FDTD update equations]{Update equations at a point depend on past values of adjacent points.}
	\label{fig:fdtd}
\end{figure} 
\item Find the magnetic field component one time-step into the future for complete spatial domain. Using equation \eqref{eq:mag}
\begin{equation}
	H_y^{q+\frac {1}{2}} \left[ m + \frac {1}{2} \right] = H_y^{q-\frac {1}{2}} \left[ m + \frac {1}{2} \right]
+ \frac {\Delta_t}{\mu\Delta_x} \left( E_z^q \left[ m+1 \right] - E_z^q \left[m\right] \right)
\label{eq:mag}
\end{equation}
\item Find the electric field component one time-step into the future for complete spatial domain. Using equation \eqref{eq:ele}
\begin{equation}
 E_z^{q+1} \left[m\right] =  E_z^q \left[m\right] + \frac {\Delta_t}{\epsilon\Delta_x}  \left( H_y^{q+\frac {1}{2}} \left[ m + \frac {1}{2} \right] - H_y^{q+\frac {1}{2}} \left[ m - \frac {1}{2} \right]  \right)
\label{eq:ele}
\end{equation}
\item Repeat step 4 and 5 until the desired duration of time.
\end{enumerate}

The equations \eqref{eq:mag} and \eqref{eq:ele} does not relate to how far energy can propagate in one time step. The maximum speed at at which energy can travel is speed of light $c = \frac {1}{\sqrt\epsilon_o\mu_o}$ hence the maximum distance is $c\Delta_t$. An important factor is Courant number $S_c =  \frac {c\Delta_t}{\Delta_x}$ which relates the maximum distance with that of distance traveled by the energy under study. The coefficients in equations \eqref{eq:mag} and \eqref{eq:ele} can be written as 
\begin{equation}
	\frac {\Delta_t}{\epsilon\Delta_x} = \frac {\eta_0}{\epsilon_r}S_c
\end{equation}
\begin{equation}
	\frac {\Delta_t}{\mu\Delta_x} = \frac {1}{\mu_r\eta_0}S_c
\end{equation}
where $\eta-0 = \sqrt\mu_0\epsilon_0$ is impedance of free space.

\begin{figure}[htbp]
	\centering
		\includegraphics[width=4in]{Pictures/1dfdtd.jpg}
		%\rule{35em}{0.5pt}
	\caption[1 dimensional FDTD Space]{A one-dimensional FDTD space showing the spatial offset between magnetic and electric fields.}
	\label{fig:1dfdtd}
\end{figure} 

%-----------------------------------
%	SUBSECTION 1.3
%-----------------------------------
\subsection{One-Dimensional FDTD simulation}
To simulate  equations \eqref{eq:mag} and \eqref{eq:ele} in MATLAB we have to keep in mind following things:
MATLAB uses integer numbers for indexes of arrays hence we use does not set an offset of $\frac{1}{2}$ instead we use same integers for indexes as well, Figure \ref{fig:fdtdpc} shows a graphical representation of FDTD algorithm with integer indexes
\begin{figure}[htbp]
	\centering
		\includegraphics[width=5in]{Pictures/fdtdpc.jpg}
		%\rule{35em}{0.5pt}
	\caption[1 dimensional FDTD Space assumed with integer indices]{A one-dimensional FDTD space showing the assumed indexes and locations of magnetic and electric component in spatial domain}
	\label{fig:fdtdpc}
\end{figure}
Keeping figure \ref{fig:fdtdpc} in mind, the two update equations become
\lstset{language=Matlab, commentstyle=\color{green!50!black}, keywordstyle=\color{blue}, stringstyle=\color{red!60!black}}
\begin{lstlisting}
for m=0:SIZE-1
	hy[m] = hy[m] + (ez[m + 1] - ez[m]) / imp0;
end
for m=1:SIZE
	ez[m] = ez[m] + (hy[m] - hy[m - 1])  * imp0;
end
\end{lstlisting}
where imp0 is impedance of free space.
The reason behind different loop start and end point is that at end nodes there are no neighboring nodes to one side. For example hy[-1] node for ez[0].
As the field is initially zero it will remain zero as there is no energy passing through it. to overcome this problem a source node is hard-coded into one of the entry of array 
\lstset{language=Matlab, commentstyle=\color{green!50!black}, keywordstyle=\color{blue}, stringstyle=\color{red!60!black}}
\begin{lstlisting}
	ez(1) = exp(-(qTime - 30) * (qTime - 30) / 100.); %update node hard-coded
\end{lstlisting}

\textbf{Results}\\
By using the program in Appendix %ref here
we get the results in figure \ref{fig:fdtdpc}.
\begin{figure}[htbp]
	\centering
		\includegraphics[width=5in]{Figures/free.jpg}
		%\rule{35em}{0.5pt}
	\caption[Simulation Result of 1 dimensional FDTD in free space]{Simulation Results of one-dimensional FDTD in free space showing both electric and magnetic component}
	\label{fig:fdtdpc}
\end{figure}
%-----------------------------------
%	SUBSECTION 1.4
%-----------------------------------
\subsection{Boundary Conditions}
Grid termination is called boundary condition because there are no neighboring values at boundary. we must use some kind of boundary condition depending upon the requirements. two type of boundary conditions are mentioned below.
%-----------------------------------
%	SUBSECTION 1.4.1
%-----------------------------------
\subsubsection{Perfect Conductor Boundary}
In program appendix % ref here
grid is terminated with a zero value of magnetic field
making it a perfect magnetic conductor (PMC) which reflects the wave completely. it also inverts the magnetic component of the wave figure \ref{fig:pmc}. In reality PMC does not exists hence to make this simulation as close to real behavior we use some other method to truncate the grid.
\begin{figure}[htbp]
	\centering
		\includegraphics[width=5in]{Figures/pmc.jpg}
		%\rule{35em}{0.5pt}
	\caption[Simulation Result of 1 dimensional FDTD in free space with Perfect magnetic conductor boundary]{Simulation Results of one-dimensional FDTD after reflecting from Perfect magnetic conductor}
	\label{fig:pmc}
\end{figure}
%-----------------------------------
%	SUBSECTION 1.4.2
%-----------------------------------
\subsubsection{Absorbing Boundary Conditions}
Perfect conductor boundary depends on the speed of propagation. It requires that at each time instant the wave also propagate by one spatial step, but with the introduction of dielectric (slab) the speed of propagation decreases resulting in unstable simulation. hence we need an advance boundary condition which is differential equation based absorbing boundary condition (ABC).
\begin{equation}
	E_z^{q+1} \left[ 0 \right] = E_z^q \left[ 1 \right] + \frac {\frac{S_c}{\sqrt\mu_r\epsilon_r}-1} {\frac{S_c}{\sqrt\mu_r\epsilon_r}+1} \left( E_z^{q+1} \left[ 1 \right] - E_z^q \left[ 0 \right]  \right)
\label{abc}
\end{equation}
Equation \eqref(abc) is absorbing boundary equation %ref here
for electric field. As it can be seen that this boundary condition also depends upon last time step value at boundary point. so we need to save that value in our program as well.
Program in appendix %ref here
implements this boundary condition.

\textbf{Results}\\
Figure \ref{fig:abc} shows Gaussian wave after passing through a denser slab with absorbing boundary condition
\begin{figure}[htbp]
	\centering
		\includegraphics[width=5in]{Figures/abc.jpg}
		%\rule{35em}{0.5pt}
	\caption[Simulation Result of 1 dimensional FDTD after passing through denser medium]{Simulation Results of one-dimensional FDTD after passing through denser medium with absorbing boundary conditions}
	\label{fig:abc}
\end{figure}

%-----------------------------------
%	SUBSECTION 1.5
%-----------------------------------
\subsection{Simulation Results}
Following results were found after running program %ref here
 this program implements FDTD algorithm in free space, denser medium slab, additive source and absorbing boundary conditions.
%-----------------------------------
%	SUBSECTION 1.5.1
%-----------------------------------
\subsubsection{Simulation Parameters}
Parameters for program %ref here
are \\
Medium 1 = Free space  $\mu=\mu_0$  $\epsilon=\epsilon_0$\\
Medium 2 = slab of denser medium $\mu=2\mu_0$  $\epsilon=2\epsilon_0$\\
Courant Number $S_c=1$\\
Source type = Additive\\
Boundary conditions = Absorbing boundary conditions
%-----------------------------------
%	SUBSECTION 1.5.2
%-----------------------------------
\subsubsection{Frequency Domain Analysis}
Frequency domain analysis compares the theoretical values with that of simulated values.
 In Frequency domain analysis spectrum of transmitted and Incident waves are plotted. 
Refractive index is also calculated which should be equal to calculated refractive index given by equation \eqref{refractiveindex}
\begin{equation}
	\eta=\sqrt\epsilon_r\mu_r
\label{refractiveindex}
\end{equation}
for medium having $\mu=2\mu_0$  $\epsilon=2\epsilon_0$ result of equation \eqref{refractiveindex} comes out to be 1.414

\begin{figure}[htbp]
	\centering
		\includegraphics[width=5in]{Figures/fft1.png}
		%\rule{35em}{0.5pt}
	\caption[Frequency Spectrum of 1D denser medium slab]{Frequency Spectrum of transmitted and reflected wave after passing through a slab of denser medium having $\mu=2\mu_0$  $\epsilon=2\epsilon_0$ }
	\label{fig:fft1}
\end{figure}
\begin{figure}[htbp]
	\centering
		\includegraphics[width=5in]{Figures/ri1.png}
		%\rule{35em}{0.5pt}
	\caption[Refractive index of different frequencies after passing through denser medium]{Refractive index of different frequencies after passing through denser medium, Red lines shows the theoretical values of refractive index at 3Ghz Frequency}
	\label{fig:ri1}
\end{figure}
From figure \ref{fig:ri1} it is clear that simulation result and theoretical results are same as calculated from equation \eqref{refractiveindex}.
%-----------------------------------
%	SECTION 2
%-----------------------------------
\section{Modeling of NIM slab }
%-----------------------------------
%	SUBSECTION 2.1
%-----------------------------------
\subsection{Limitation of FDTD}
The standard FDTD does not work for negative values of permittivity and permability. Reson behind this is Courant stability criterion. hence a NIM object can not be modeled using standard FDTD algorithm. Drude dispersive model or Lorentz model are used to implement NIM object. These models introduce frequency of operation $(\omega)$ into update equations.
%-----------------------------------
%	SUBSECTION 2.2
%-----------------------------------
\subsection{The Drudes Model}
Ideally permeability and permitivity of a material remain constant for all values of operating frequencies. However in reality Speed of electromagnetic waves changes with frequency and there is also loss due to particle collisions inside material. A material is called dispersive if it's permability or permittivity is frequency depandent.
Paual Drude proposed a model of transport properties in materials in 1990\citep{drude}.
\begin{equation}
	M\frac{d^2x}{dt^2} = QE(t) - Mg\frac{dx}{dt}
\label{drudee1}
\end{equation}
Equation \eqref{drudee1} is a Drude's 2nd order differential equation that relates to kinetic energy in moving charges under an electric field. The relative Permittivity in Drudes model is given by 
\begin{equation}
\hat{\epsilon_r} (w) = \epsilon_{\infty}- \frac{\omega_p^2}{\omega^2 - jg\omega}
\label{drudee2}
\end{equation}
setting $g=0$ and $ \epsilon_{\infty} = 1$ in equation \eqref{drudee2}, value of permittivity comes out to be negative for $\frac{\omega}{\omega_p} > 1$ (figure \ref{drude1})
\begin{figure}[htbp]
	\centering
		\includegraphics[width=5in]{Figures/drude1.jpg}
		%\rule{35em}{0.5pt}
	\caption[Permittivity in Drudes model for different frequencies]{Relative Permittivity plotted against $\frac{\omega}{\omega_p}$ for $g=0$ and $ \epsilon_{\infty} = 1$ }
	\label{drude1}
\end{figure}

%----------------------------------------------------------------------------------------
%	SUB SUB SECTION 2.2.1
%----------------------------------------------------------------------------------------

\subsubsection{Drudes Algorithm}
Auxiliary update equations for electric and magnetic components in Drudes model are give by \eqref{drudemag} and \eqref{drudeele}.
\begin{equation}
%\begin{IEEEeqnarray}{lCr}
	H_y^{n+1} = a_m \left( B_y^{n+1} - 2B_y^n + B_y^{n-1} \right) + b_m \left( B_y^(n+1) - B_y^{n-1}  \right)+ 
c_m \left( 2H_y^n-H_y^{n-1} \right) + d_m \left( 2H_y^n + H_y^{n-1} \right) +e_m H_y^{n-1}
\label{drudemag}
%\end{IEEEeqnarray}
\end{equation}
\begin{equation}
%\begin{IEEEeqnarray}{lCr}
	E_z^{n+1} = a_e \left( D_z^{n+1} - 2D_z^n + D_z^{n-1} \right) + b_e \left( D_z^(n+1) - D_z^{n-1}  \right)+ 
c_e \left( 2E_z^n-E_z^{n-1} \right) + d_e \left( 2E_z^n + E_z^{n-1} \right) +e_e E_z^{n-1}
\label{drudeele}
%\end{IEEEeqnarray}
\end{equation}
Update Equation for wave propagtion in $-x$ direction are given by 
\begin{equation}
	B_y^{n+1}(k)=  B_y^{n} (k) + \frac {\Delta t}{\Delta z} \left( E_z^n (k) - E_z^n (k+1) \right)
\label{drudeby}
\end{equation}
\begin{equation}
	D_z^{n+1}(k)=  D_z^{n} (k) + \frac {\Delta t}{\Delta z} \left( H_y^n (k-1) - H_y^n (k) \right)
\label{drudedz}
\end{equation}

%-----------------------------------
%	SUBSECTION 2.3
%-----------------------------------
\subsection{Simulation of 1D DNG Slab}
%-----------------------------------
%	SUBSECTION 2.3.1
%-----------------------------------
\subsection{Problem Specification}
An EM wave is incident on a slab with negative permittivity and permeability ($\epsilon_r= -1$ $\mu_r= -1$) at frequency of 3Ghz for Sinusoidal wave. Guassian wave is also applied to same slab for wide range frequancy analysis. MATLAB code is available in 
Appendix %ref here
%-----------------------------------
%	SUBSECTION 2.3.2
%-----------------------------------
\subsection{Simulation Parameters}
Frequency of Opertation $f=3 GHz$, $S_c=1$, in order to get $\epsilon_r= -1$ $\mu_r= -1$ plasma frequency needs to be $\omega_{pm}^2 = \omega_{pe}^2 = 2 \times (2\pi f_0)^2$ with $\epsilon_\infty = \mu_\infty = 1$.Absorbing Boundary Conditions are applied at field boundary.
%-----------------------------------
%	SUBSECTION 2.3.3
%-----------------------------------
\subsection{Simulation Results}
Following Results are optained from simulation of 1D DNG slab
\textbf{Guassian Source}\\
\begin{figure}[htbp]
	\centering
		\includegraphics[width=5in]{Figures/drude3.jpg}
		%\rule{35em}{0.5pt}
	\caption[Guassian Pulse after Passing through DNG slab]{conversion of Guassian pulse into multiple wave forms of different frequencies after passing thorugh dispersive DNG slab}
	\label{drude3}
\end{figure}

\begin{figure}[htbp]
	\centering
		\includegraphics[width=5in]{Figures/drude5.jpg}
		%\rule{35em}{0.5pt}
	\caption[Reflected and Transmitted wave's Frequency Spectrum for 1D DNG simulation]{Reflected and Transmitted wave's Frequency Spectrum for 1D DNG simulation with Guassian wave as source}
	\label{drude5}
\end{figure}
\begin{figure}[htbp]
	\centering
		\includegraphics[width=5in]{Figures/drude4.jpg}
		%\rule{35em}{0.5pt}
	\caption[Refractive Index vs Frequency for 1D DNG slab]{Refractive indexes(Y-axis) Vs Frequency Graphs(X-axis) for 1D DNG slab, Black dashed line shows therotical values, Blue line shows simulation results and calculated values, Grey lines mark the Frequency under study i.e 3GHz }
	\label{drude4}
\end{figure}
%----------------------------------------------------------------------------------------
%	SECTION 3
%----------------------------------------------------------------------------------------

\section{Implementation in C++}

test section tasdlasdj as
asdsajdlsajdsa
asdasda\\
asdsadasd\\
asdasd\\
%-----------------------------------
%	SECTION 4
%-----------------------------------
\section{Implementation on GPU }

%-----------------------------------
%	SUBSECTION 4.1
%-----------------------------------
\subsection{AMD 6770M}
 
%-----------------------------------
%	SUBSECTION 4.2
%-----------------------------------
\subsection{OpenCL}


